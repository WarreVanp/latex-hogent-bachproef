% !TeX program = xelatex
% !BIB program = biber

%==============================================================================
% Paper Research Methods: onderzoeksvoorstel
%==============================================================================

\PassOptionsToPackage{backend=biber,style=apa,sortcites=true,uniquename=false,maxbibnames=99}{biblatex}

\documentclass{hogent-article}

% --- STRUCTUUR/OPMAAK ---
\usepackage[protrusion=true,expansion=false]{microtype}
\emergencystretch=2em
\sloppy
\raggedbottom

\usepackage[dutch]{babel}
\usepackage{csquotes}
\DeclareLanguageMapping{dutch}{dutch-apa}

\addbibresource{voorstel.bib}

\usepackage{graphicx}
\graphicspath{{.}{../graphics/}} % Pad aangepast naar het hogere niveau (../graphics/)

\usepackage{caption}
\captionsetup{font=small,labelfont=bf}

\usepackage[section]{placeins}
\usepackage{float}

\usepackage{xurl}

\AtBeginDocument{\hypersetup{hidelinks}}

\usepackage{enumitem}
\setlist{topsep=3pt,itemsep=2pt,parsep=0pt,partopsep=0pt}

\clubpenalty=10000
\widowpenalty=10000
\displaywidowpenalty=10000

\setcounter{tocdepth}{2}

\hyphenation{
 klant-vragen
 gebruikers-account-analyse
 klanten-ser-vice
 door-ver-wij-zingen
 ver-wer-king
 ge-au-to-ma-ti-seer-de
}

% --- INFO OVER OPDRACHT ------------------------------------------------------
% Informatie over de opleiding, het vak en soort opdracht
\studyprogramme{Professionele bachelor toegepaste informatica}
\course{Research Methods}
\assignmenttype{Paper: Onderzoeksvoorstel}
\academicyear{2024--2025}

% TODO (fase 1): Werktitel
\title{Hoe kan een AI-gestuurd systeem de relatie analyseren tussen schappositie (kinderooghoogte, volwassenenooghoogte, hoog) 
en voedingswaarde (Nutri-Score A-E) van plant-based vleesvervangers en ontbijtgranen in Belgische supermarkten?}

% TODO (fase 1): Studentnaam en emailadres invullen
\author{Warre Vanpaemel}
\email{warre.vanpaemel@student.hogent.be}

% TODO (fase 1): Medestudent    
% Schrijf je het voorstel in samenwerking met een medestudent? Geef dan de naam
% en emailadres hier. Als je het voorstel alleen schrijft, verwijder dan deze
% regels of zet ze in commentaar.

% TODO (fase 1): Geef hier de link naar jullie Github-repository
\projectrepo{https://github.com/WarreVanp/latex-hogent-bachproef}

\specialisation{AI \& Data Engineering}
\keywords{AI, klantenservice, gebruikersanalyse, automatisering, NLP}

\begin{document}

\begin{abstract}
Dit onderzoek pakt het gebrek aan geautomatiseerde en objectieve methoden aan om de relatie 
tussen \textbf{schappositie} en \textbf{voedingswaarde} (Nutri-Score $A-E$) van plant-based producten
in Belgische supermarkten te analyseren. Voortbouwend op de Vlaamse eiwitstrategie 
(die streeft naar $60\%$ plantaardige eiwitconsumptie) en het feit dat ooghoogte het aankoopgedrag beïnvloedt, 
is de centrale vraag hoe een AI-gestuurd systeem deze correlatie kan vaststellen. 
De doelstelling is de ontwikkeling van een end-to-end AI-pipeline, waarbij \textbf{YOLOv8} object detection en 
OCR (Optical Character Recognition) worden ingezet om producten en Nutri-Scores te herkennen op $100-150$ geannoteerde schapfoto's. 
Na classificatie in drie verticale zones (kinderooghoogte, volwassenenooghoogte, hoog), zal het systeem statistische correlaties tonen. 
De verwachte uitkomst is een systeem met minimaal $85\%$ detectienauwkeurigheid dat bewijs levert over 
de positionering van ongezonde producten (Nutri-Score $D/E$) op kinderooghoogte. Dit biedt supermarktketens een schaalbare monitoring-tool 
en geeft beleidsmakers objectieve data om ethische marketingrichtlijnen te evalueren.
\end{abstract}

\tableofcontents

\bigskip

% TODO: EP3

\section{Introductie}

Dit bacheloronderzoek richt zich op het ontwikkelen van een \textbf{AI-gestuurd computer systeem} 
om de schappositie van \textbf{plant-based vleesvervangers en ontbijtgranen} in Belgische supermarkten objectief
te analyseren en de relatie te onderzoeken met hun \textbf{voedingswaarde} (Nutri-Score $A-E$). De Vlaamse eiwitstrategie
2021--2030 streeft naar een verschuiving waarbij minstens $60\%$ van de eiwitinname plantaardig is. Tegelijkertijd 
breiden Belgische supermarktketens hun plant-based aanbod uit. Voor beleidsmakers en category managers is het cruciaal 
om inzicht te hebben in hoe producten gepositioneerd worden en of consumenten hierbij een voorkeur voor gezondere keuzes 
krijgen, bijvoorbeeld door gezonde producten op ooghoogte te plaatsen.

\subsection{Probleemstelling}

Handmatige analyses van schappen zijn tijdrovend, duur en subjectief. Er bestaat geen schaalbare methode om systematisch te meten welke producten op welke hoogte en met welke zichtbaarheid aanwezig zijn, noch hoe dit samenhangt met de voedingswaarde. Dit vormt een belemmering voor:
\begin{itemize}
\item Category managers die schapindeling willen optimaliseren
\item Beleidsmakers die de effectiviteit van de Vlaamse eiwitstrategie willen monitoren
\item Retail consultants die evidence-based advies willen geven over schapindeling
\end{itemize}

\subsection{Centrale onderzoeksvraag}

\begin{quote}
Hoe kan een AI-gestuurd systeem automatisch de relatie analyseren tussen schappositie 
(kinderooghoogte, volwassenenooghoogte, hoog) en voedingswaarde (Nutri-Score $A-E$) van plant-based vleesvervangers 
en ontbijtgranen in Belgische supermarkten?
\end{quote}

\subsection{Doelstellingen}

De doelstellingen van dit onderzoek zijn:
\begin{enumerate}
\item Ontwikkelen van een end-to-end pipeline die producten detecteert, Nutri-Score herkent, verticale zones classificeert en correlaties berekent.
\item Opbouwen van een gelabelde dataset van Belgische schapfoto's met annotaties voor product, label en schapzone.
\item Optimaliseren van object-detection en label-recognition modellen voor supermarktcondities.
\item Valideren van nauwkeurigheid tegen handmatige metingen.
\item Levering van praktische visualisaties en KPI's die bruikbaar zijn voor retailers en beleidsmakers.
\end{enumerate}

\section{State-of-the-Art}

\subsection{Context en relevantie}

Recente literatuur toont een toename van plant-based producten in Europese schappen 
en een groeiende vraag naar objectieve monitoringtools. Productherkenning in retail combineert 
vaak \textbf{beeld- en tekstinformatie} (multimodaal) om fijne onderscheidingen tussen producten te maken.

\subsection{Belangrijke bevindingen}

\begin{itemize}
\item Multimodale benaderingen (beeld + OCR) verhogen de classificatie-robustheid in grocery recognition~\cite{Pettersson2024, Patel2025}.
\item De \textbf{YOLO-familie} (YOLOv8) biedt een optimale trade-off tussen snelheid en nauwkeurigheid, geschikt voor real-time detectie~\cite{Redmon2024}.
\item Dense packing en kleine objecten blijven uitdagingen; technieken zoals tiling en aangepaste non-max suppression helpen hier~\cite{Goldman2019}.
\item Nutri-Score en andere labels kunnen betrouwbaar worden herkend met OCR gecombineerd met template matching of kleine CNN-classifiers~\cite{Patel2025}.
\item Spatial analysis (facing count, shelf share, eye-level premium) wordt meestal handmatig uitgevoerd; automatische methodes zijn schaars~\cite{Goldman2019}.
\end{itemize}

\subsection{Gelijksoortige onderzoeken}

Er zijn studies op grocery product recognition datasets, maar weinig focussen 
op \textbf{plant-based producten} gecombineerd met Nutri-Score en verticale schapposities in Belgische context. 
Bestaande datasets missen vaak lokale merken en het specifieke labelgebruik. Dit onderzoek vult deze gebrekken aan met:
\begin{itemize}
\item Regionale dataset van Belgische filialen.
\item Focus op correlatie schappositie vs Nutri-Score.
\item Technische diepgang: benchmarking van detectors, spatial analysis, facing counts.
\end{itemize}

\subsection{Toegevoegde waarde}

Dit onderzoek levert:
\begin{itemize}
\item Een reproduceerbare, multimodale AI-pipeline voor detectie, labelherkenning en positioneringsanalyse.
\item Statistische evaluatie van de associatie tussen verticale schappositie en voedingswaarde.
\item Praktische visualisaties en KPI's voor category management en beleidsanalyse.
\end{itemize}

\section{Methodologie}

\subsection{Overzicht architectuur}

De pipeline bestaat uit:
\begin{enumerate}
\item \textbf{Data-acquisitie:} $100-150$ hoge-res foto’s van schappen in $3-5$ filialen; metadata (filiaal, schap, opnamehoek, tijdstip).
\item \textbf{Annotatie:} Bounding boxes voor producten, labels (Nutri-Score, Bio, Vegan), productcategorie, schapzone en facing ID met CVAT/LabelImg; datasetbeheer via Roboflow.
\item \textbf{Object Detection:} YOLOv8 (primary), Faster R-CNN (benchmark); fine-tuning, augmentaties met Albumentations, tiling-strategie.
\item \textbf{Label Recognition:} EasyOCR + Tesseract; template matching of kleine CNN-classifier voor Nutri-Score badges; fallback logica met product lookup.
\item \textbf{Spatial Analysis:} Verticale zoneclassificatie, facing counting, shelf share berekening, label zichtbaarheidscore.
\item \textbf{Analyse:} Statistische tests (chi-kwadraat, ANOVA/Kruskal-Wallis, regressie), validatie tegen handmatige ground-truth.
\item \textbf{Visualisatie:} Annotated images, heatmaps, KPI dashboard (Streamlit/Plotly Dash), export CSV/PDF.
\end{enumerate}

\subsection{Tools}

\textbf{Hardware:} GPU-workstation (min. $16$ GB VRAM, b.v. RTX $3080$), smartphone/mirrorless camera met referentiekaart.  

\textbf{Software:} Python $3.9+$, PyTorch/TensorFlow, Ultralytics YOLOv8, OpenCV, Albumentations, EasyOCR/Tesseract, pandas/numpy/scikit-learn, matplotlib/plotly, Streamlit/Dash, Docker, Git.  

\subsection{Validatie en evaluatie}

\begin{itemize}
\item Detection: mAP@0.5, mAP@0.5:0.95, Precision, Recall.
\item Labelherkenning: Accuracy, F1 per labeltype.
\item Spatial metrics: afwijking schapruimte, facing count error.
\item Statistische analyse: p-waarden, odds ratios, betrouwbaarheidsintervallen.
\end{itemize}

\subsection{Tijdsschatting (modulair)}

\begin{enumerate}
\item Data-acquisitie en annotatie: $4-5$ weken.
\item Modelontwikkeling (YOLOv8 detectie, label recognition, augmentatie): $5$ weken.
\item Spatial analysis en validatie: $3$ weken.
\item Visualisatie en dashboard: $2$ weken.
\item Rapportage en conclusie: $2$ weken.
\end{enumerate}

\section{Verwachte resultaten}

\begin{itemize}
\item Reproduceerbare pipeline voor productdetectie, Nutri-Score herkenning en schapanalyse.
\item Benchmarkresultaten YOLOv8 vs Faster R-CNN: nauwkeurigheid vs inference tijd.
\item Statistische correlatieanalyse: verband verticale schappositie vs Nutri-Score, incl. odds ratios.
\item KPI-dashboard voor category management en beleidsmonitoring.
\end{itemize}

\subsection{Impact voor doelgroep}

\begin{itemize}
\item \textbf{Category managers:} objectieve data voor schapoptimalisatie, inzicht in positioning van gezonde producten.
\item \textbf{Beleidsmakers:} empirisch bewijs voor regulering/ethische marketing en evaluatie Vlaamse eiwitstrategie.
\item \textbf{Retail consultants:} schaalbare monitoringtool voor advies en A/B testen van schaplayouts.
\end{itemize}

\section{Conclusie}

Dit onderzoek ontwikkelt een technisch robuuste, multimodale AI-pipeline voor \textbf{detectie, labelherkenning en spatial analysis} van plant-based producten in Belgische supermarkten. Het combineert state-of-the-art object detection (YOLOv8), OCR/template-based labelherkenning en nauwkeurige spatial metrics (facing counts, shelf share) op een gelabelde, regionale dataset. Verwachte resultaten omvatten reproduceerbare KPI’s en visualisaties, statistische evaluaties van schappositie vs Nutri-Score, en praktische tools voor retailers en beleidsmakers. Hiermee levert het onderzoek zowel wetenschappelijke als maatschappelijke meerwaarde.

\printbibliography[heading=bibintoc]

\end{document}