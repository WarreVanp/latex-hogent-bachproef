% !TeX program = xelatex
% !BIB program = biber

%==============================================================================
% Paper Research Methods: onderzoeksvoorstel
%==============================================================================

\PassOptionsToPackage{backend=biber,style=apa,sortcites=true,uniquename=false,maxbibnames=99}{biblatex}

\documentclass{hogent-article}

% --- STRUCTUUR/OPMAAK ---
\usepackage[protrusion=true,expansion=false]{microtype}
\emergencystretch=2em
\sloppy
\raggedbottom

\usepackage[dutch]{babel}
\usepackage{csquotes}
\DeclareLanguageMapping{dutch}{dutch-apa}

\addbibresource{voorstel.bib}

\usepackage{graphicx}
\graphicspath{{.}{../graphics/}}

\usepackage{caption}
\captionsetup{font=small,labelfont=bf}

\usepackage[section]{placeins}
\usepackage{float}

\usepackage{xurl}

\AtBeginDocument{\hypersetup{hidelinks}}

\usepackage{enumitem}
\setlist{topsep=3pt,itemsep=2pt,parsep=0pt,partopsep=0pt}

\clubpenalty=10000
\widowpenalty=10000
\displaywidowpenalty=10000

\setcounter{tocdepth}{2}

\hyphenation{
 klant-vragen
 gebruikers-account-analyse
 klanten-ser-vice
 door-ver-wij-zingen
 ver-wer-king
 ge-au-to-ma-ti-seer-de
}

% --- INFO OVER OPDRACHT ------------------------------------------------------
\studyprogramme{Professionele bachelor toegepaste informatica}
\course{Research Methods}
\assignmenttype{Paper: Onderzoeksvoorstel}
\academicyear{2024--2025}

\title{Hoe kan een AI-gestuurd systeem de relatie analyseren tussen schappositie 
en voedingswaarde van plant-based vleesvervangers en ontbijtgranen in Belgische supermarkten?}

\author{Warre Vanpaemel}
\email{warre.vanpaemel@student.hogent.be}

\projectrepo{https://github.com/WarreVanp/latex-hogent-bachproef}

\specialisation{AI \& Data Engineering}
\keywords{Computer Vision, Object Detection, YOLOv8, Nutri-Score Detection, 
Shelf Positioning, Retail Analytics, Plant-Based Food, Spatial Analysis, OpenCV}

\begin{document}

\begin{abstract}
In het kader van de Vlaamse eiwitstrategie 2021-2030, die streeft naar 60\% plantaardige 
eiwitconsumptie, en de toenemende beschikbaarheid van plant-based vleesvervangers en 
gezondere ontbijtgranen in Belgische supermarkten, rijst de vraag of de schappositie van 
deze producten correleert met hun voedingswaarde. Uit retailonderzoek blijkt dat producten 
op ooghoogte, en specifiek op kinderooghoogte, een significante invloed hebben op 
aankoopgedrag, maar het ontbreekt aan objectieve, geautomatiseerde methoden om de relatie 
tussen schappositie en gezondheidswaarde (Nutri-Score) systematisch te analyseren. Dit 
onderzoek beantwoordt de vraag: hoe kan een AI-gestuurd computer vision systeem de relatie 
analyseren tussen schappositie (kinderooghoogte, volwassenenooghoogte, hoog) en 
voedingswaarde (Nutri-Score A-E) van plant-based vleesvervangers en ontbijtgranen in 
Belgische supermarkten? De doelstelling is het ontwikkelen van een end-to-end AI-pipeline 
die automatisch producten detecteert via YOLOv8 object detection, Nutri-Score labels herkent 
door middel van template matching en OCR, de verticale schappositie classificeert in drie 
hoogtzones, en statistische correlaties visualiseert. De methodologie omvat het fotograferen 
en annoteren van 100-150 schapfoto's in 3-5 supermarktfilialen, het trainen van deep learning 
modellen op deze dataset, het implementeren van spatial analysis algoritmes voor 
hoogteclassificatie, en het valideren van de AI-nauwkeurigheid tegen handmatige metingen. 
Verwacht wordt dat het systeem een detectienauwkeurigheid van minimaal 85\% bereikt en 
aantoont of er een significante correlatie bestaat tussen schappositie en Nutri-Score, 
specifiek of ongezonde producten (Nutri-Score D/E) disproportioneel op kinderooghoogte 
gepositioneerd worden. De meerwaarde voor supermarktketens ligt in een schaalbare 
monitoring-tool voor category management en naleving van ethische marketing-richtlijnen, 
terwijl beleidsmakers objectieve data krijgen om de effectiviteit van de Vlaamse 
eiwitstrategie te evalueren en eventuele regulering van schappositie voor ongezonde 
producten te onderbouwen.
\end{abstract}

\tableofcontents

\bigskip

\section{Introductie}
\label{sec:introductie}

Dit bacheloronderzoek richt zich op het ontwikkelen van een \textbf{AI-gestuurd computer 
vision systeem} om de schappositie van \textbf{plant-based vleesvervangers en ontbijtgranen} 
in Belgische supermarkten objectief te analyseren en de relatie te onderzoeken met hun 
\textbf{voedingswaarde} (Nutri-Score $A-E$). De Vlaamse eiwitstrategie 2021--2030 streeft 
naar een verschuiving waarbij minstens $60\%$ van de eiwitinname plantaardig is 
\autocite{VlaanderenEiwitstrategie2023}. Sinds de start van deze strategie staan er inmiddels 
$113$ eiwitacties en $90$ onderzoeksprojecten op de teller \autocite{VlaanderenEiwitVoortgang2025}. 

Tegelijkertijd breiden Belgische supermarktketens hun plant-based aanbod uit: sinds mei 2024 
hebben retailers zoals Colruyt, Lidl, Albert Heijn en Aldi blended meat-producten 
geïntroduceerd met groot succes \autocite{GreenQueenBlendedMeat2024}, terwijl Colruyt Group 
zijn Boni Plan't submerk lanceerde met meer dan $100$ vleesvrije producten 
\autocite{VegconomistColruyt2024}. Voor beleidsmakers en category managers is het cruciaal 
om inzicht te hebben in hoe producten gepositioneerd worden en of consumenten hierbij een 
voorkeur voor gezondere keuzes krijgen, bijvoorbeeld door gezonde producten op ooghoogte 
te plaatsen.

\subsection{Maatschappelijke context}
\label{subsec:maatschappelijke-context}

De Europese plant-based vleesmarkt was USD $2,47$ miljard in 2024 en zal naar verwachting 
groeien tot USD $9,54$ miljard tegen 2033, met een jaarlijkse groei van $16,2\%$ 
\autocite{ResearchMarketsEurope2025}. Onderzoek van het Smart Protein Project toont aan dat 
flexitariërs $30\%$ van Europa's bevolking uitmaken, oplopend tot $40\%$ wanneer vegetariërs 
en pescatariërs worden meegerekend \autocite{ResearchMarketsEurope2025}. 

In België actualiseerde de overheid in 2025 haar voedingsrichtlijnen met grotere nadruk op 
plantaardig voedsel, waarbij aanbevolen wordt het consumptie van rood vlees te beperken tot 
maximaal $300$g per week \autocite{FoodNavigatorBelgium2025}. Longitudinaal onderzoek in 
België toont een stijging van consumententevredenheid met plantaardige alternatieven van 
$44\%$ in 2019 naar meer dan $50\%$ in 2020 \autocite{BryantSanctorumc2021}, en uit recent 
onderzoek blijkt dat $67\%$ van de Belgen bereid is meer te betalen voor duurzame producten 
\autocite{MVOVlaanderenSustainapoll}.

\subsection{Probleemstelling}
\label{subsec:probleemstelling}

Handmatige analyses van schappen zijn tijdrovend, duur en subjectief. Er bestaat geen 
schaalbare methode om systematisch te meten welke producten op welke hoogte en met welke 
zichtbaarheid aanwezig zijn, noch hoe dit samenhangt met de voedingswaarde. Dit vormt een 
belemmering voor:

\begin{itemize}
\item Category managers die schapindeling willen optimaliseren
\item Beleidsmakers die de effectiviteit van de Vlaamse eiwitstrategie willen monitoren
\item Retail consultants die evidence-based advies willen geven over schapindeling
\end{itemize}

\subsection{Centrale onderzoeksvraag}
\label{subsec:onderzoeksvraag}

\begin{quote}
Hoe kan een AI-gestuurd vision-systeem automatisch de relatie analyseren tussen schappositie 
(kinderooghoogte, volwassenenooghoogte, hoog) en voedingswaarde (Nutri-Score $A-E$) van 
plant-based vleesvervangers en ontbijtgranen in Belgische supermarkten?
\end{quote}

\subsection{Doelstellingen}
\label{subsec:doelstellingen}

De doelstellingen van dit onderzoek zijn:

\begin{enumerate}
\item Ontwikkelen van een end-to-end pipeline die producten detecteert, Nutri-Score herkent, 
verticale zones classificeert en correlaties berekent.
\item Opbouwen van een gelabelde dataset van Belgische schapfoto's met annotaties voor 
product, label en schapzone.
\item Het \textbf{benchmarking} en \textbf{fine-tunen} van de geselecteerde deep learning 
modellen (waaronder YOLOv8 en kleine CNN-classifiers) om robuuste product- en 
Nutri-Score-herkenning te garanderen onder diverse supermarktcondities.
\item Het \textbf{valideren} van de nauwkeurigheid en robuustheid van de gehele AI-pipeline 
aan de hand van een onafhankelijke set handmatige referentiedata en het vaststellen van de 
prestatie-indicatoren (zoals mAP en Nutri-Score F1-score).
\item Levering van praktische visualisaties en KPI's die bruikbaar zijn voor retailers en 
beleidsmakers.
\end{enumerate}

\section{State-of-the-Art}
\label{sec:state-of-the-art}

\subsection{Context en technologische basis}
\label{subsec:context}

Recente literatuur toont een duidelijke toename van plant-based producten in Europese 
schappen, wat de behoefte aan geavanceerde, objectieve monitoringtools onderstreept. De 
Europese plant-based vleesmarkt kende een gestage groei over de afgelopen vijf jaar, met 
een verwachte jaarlijkse groei van $6\%$ in zowel verkoop als volume tot 2027 
\autocite{InnovaMarketInsights2025}. 

Productherkenning in de retailomgeving vereist een \textbf{multimodale benadering} waarbij 
zowel visuele kenmerken als tekstuele informatie worden gecombineerd om nauwkeurige 
onderscheidingen tussen \textbf{fine-grained} productcategorieën te maken 
\autocite{Pettersson2024, Patel2025}. Dit is essentieel omdat \textbf{dense packing}, kleine 
objecten en onregelmatige belichting in schappen grote uitdagingen vormen voor pure 
vision-systemen \autocite{Goldman2019}.

Het European Academies' Science Advisory Council (EASAC) publiceerde in 2025 een 
wetenschappelijk rapport dat de nutritionele, ecologische, technologische en regelgevende 
dimensies van vleesvervangers analyseert en benadrukt dat Europa nu moet handelen om voorop 
te blijven lopen in de eiwittransitie \autocite{EASACMeatAlternatives2025}.

\subsection{Computer vision technieken}
\label{subsec:cv-technieken}

Voor de primaire detectietaak is de \textbf{YOLO-familie} (You Only Look Once), met name de 
nieuwere versie YOLOv8, de industriestandaard geworden. Deze biedt een optimale trade-off 
tussen snelheid en nauwkeurigheid, wat cruciaal is voor het real-time of batchgewijs 
verwerken van een groot volume aan schapfoto's \autocite{Redmon2024}. 

De robuuste herkenning van labels zoals Nutri-Score vereist verder een gespecialiseerde 
aanpak. Dit gebeurt door het combineren van generieke OCR-oplossingen met template matching 
of kleine CNN-classifiers die specifiek getraind zijn op de badges \autocite{Patel2025}. 
Nutri-Score labels kunnen betrouwbaar worden herkend dankzij gestandaardiseerd visueel 
ontwerp volgens de officiële grafische charter \autocite{SantePubliqueNutriscore2024}.

Terwijl \textbf{spatial analysis} (zoals facing counts en \textbf{eye-level premium}) 
momenteel meestal handmatig wordt uitgevoerd, zal dit onderzoek de schaarse methodes voor 
geautomatiseerde verticale positionering verder ontwikkelen \autocite{Goldman2019}.

\subsection{Onderzoekslacunes en toegevoegde waarde}
\label{subsec:toegevoegde-waarde}

Hoewel er diverse studies bestaan over algemene \textbf{grocery product recognition} 
datasets, is er een duidelijk gebrek aan onderzoek dat zich specifiek richt op 
\textbf{plant-based producten} in combinatie met de analyse van Nutri-Score en verticale 
schapposities binnen de Belgische retailcontext. Bestaande datasets missen vaak lokale 
merken (zoals Boni Plan't, Delhaize Taste of Tomorrow) en het specifieke labelgebruik in de 
Belgische markt. 

Dit onderzoek vult deze gebrekken aan met:
\begin{itemize}
\item Regionale dataset van Belgische filialen met lokale huismerken.
\item Focus op correlatie schappositie vs Nutri-Score, wat direct aansluit bij 
beleidsdoelstellingen van de Vlaamse eiwitstrategie.
\item Technische diepgang: benchmarking van detectors (YOLOv8 vs Faster R-CNN), spatial 
analysis, facing counts.
\item Praktische relevantie voor de groeiende Europese plant-based markt (CAGR $16,2\%$ tot 
2033) \autocite{ResearchMarketsEurope2025}.
\end{itemize}

De toegevoegde waarde ligt in de levering van een reproduceerbare, multimodale AI-pipeline 
en een statistische evaluatie van de associatie tussen verticale schappositie en 
voedingswaarde \autocite{VlaanderenEiwitstrategie2023}, resulterend in praktische KPI's die 
bruikbaar zijn voor zowel category management als beleidsanalyse 
\autocite{InnovaMarketInsights2025, EASACMeatAlternatives2025}.

\section{Methodologie}
\label{sec:methodologie}

De kern van de methodologie omvat zeven stappen, van data-acquisitie tot de uiteindelijke 
statistische analyse.

\subsection{Data-acquisitie en annotatie}
\label{subsec:data}

De eerste fase richt zich op data. Dit begint met de \textbf{Data-acquisitie}, waarbij 
$100-150$ hoge-resolutie foto's worden verzameld van relevante schappen in $3-5$ 
verschillende supermarkten om voldoende variatie in belichting en lay-out te garanderen. 
Essentiële metadata, zoals filiaal, schaplocatie en tijdstip, wordt vastgelegd. 

Dit wordt onmiddellijk gevolgd door een intensieve \textbf{Annotatie}-fase. Bounding boxes 
worden handmatig getrokken voor elk product, Nutri-Score label en andere relevante labels 
(Bio, Vegan) met behulp van tools zoals CVAT. Cruciaal hierbij is het labelen van de 
productcategorie, de schapzone en een unieke \textbf{facing ID} per product, waarna de 
dataset via platforms als Roboflow wordt beheerd.

\subsection{Modelontwikkeling}
\label{subsec:modelontwikkeling}

Zodra de dataset is gelabeld, verschuift de focus naar modelontwikkeling. De \textbf{Object 
Detection}-stap gebruikt \textbf{YOLOv8} als primaire detector, aangevuld met een benchmark 
tegen een traditionele detector zoals Faster R-CNN \autocite{RenFasterRCNN2015}. De modellen 
worden geoptimaliseerd door \textit{fine-tuning} op de eigen dataset, het gebruik van 
\textit{augmentaties} via Albumentations, en de implementatie van een \textit{tiling-strategie} 
om kleine objecten beter te kunnen detecteren. 

Parallel daaraan vindt de \textbf{Label Recognition} plaats. Dit vereist een multi-staps 
benadering: eerst tekstextractie via generieke OCR (EasyOCR/Tesseract), gevolgd door 
gespecialiseerde template matching of een kleine CNN-classifier om de Nutri-Score badges te 
lokaliseren en classificeren.

\subsection{Spatial analysis en validatie}
\label{subsec:spatial-analysis}

De output van de detectie- en herkenningsmodellen voedt vervolgens de \textbf{Spatial 
Analysis}. Dit is de meest kritieke fase van de analyse: op basis van de gedetecteerde 
pixelcoördinaten wordt de verticale zone geclassificeerd (kinderooghoogte, volwassene, hoog). 
Hieruit worden nauwkeurige \textbf{facing counts}, het \textbf{shelf share} (percentage 
schapruimte), en een label zichtbaarheidsscore berekend.

\subsection{Analyse en visualisatie}
\label{subsec:analyse}

De laatste fasen omvatten de \textbf{Analyse} en de presentatie. Voor de analyse worden de 
verkregen metingen statistisch getoetst (chi-kwadraat, ANOVA, regressie) om de correlatie 
tussen schaphoogte en voedingswaarde te bepalen. De resultaten van de AI-pipeline worden 
gevalideerd tegen onafhankelijke handmatige \textbf{ground-truth} metingen om de 
nauwkeurigheid vast te stellen. 

Tot slot volgt de \textbf{Visualisatie}: de resultaten worden gepresenteerd middels 
\textbf{annotated images}, heatmaps en een interactief KPI-dashboard, gebouwd met Streamlit 
of Plotly Dash, om de data direct bruikbaar te maken voor de doelgroep.

\subsection{Tijdsschatting}
\label{subsec:tijdsschatting}

\begin{enumerate}
\item Data-acquisitie en annotatie: $4$ weken.
\item Modelontwikkeling (YOLOv8 detectie, label recognition, augmentatie): $4$ weken.
\item Spatial analysis en validatie: $2$ weken.
\item Visualisatie en dashboard: $1$ week.
\item Rapportage en conclusie: $2$ weken.
\end{enumerate}

\section{Verwachte resultaten}
\label{sec:verwachte-resultaten}

De belangrijkste verwachte resultaten zijn de oplevering van:

\begin{itemize}
\item Een volledig \textbf{reproduceerbare AI-pipeline} voor productdetectie, Nutri-Score 
herkenning en schapanalyse, toepasbaar op de groeiende Europese plant-based markt (verwachte 
groei tot USD $9,54$ miljard tegen 2033) \autocite{ResearchMarketsEurope2025}.
\item Gedetailleerde \textbf{benchmarkresultaten} die de nauwkeurigheid versus de 
inferentietijd van YOLOv8 en eventuele andere geteste modellen (zoals Faster R-CNN) 
documenteren \autocite{Redmon2024, RenFasterRCNN2015}.
\item Een \textbf{statistische correlatieanalyse} die het verband tussen verticale 
schappositie en Nutri-Score empirisch onderbouwt, inclusief odds ratios en 
betrouwbaarheidsintervallen.
\item Een praktisch en bruikbaar \textbf{KPI-dashboard} dat dient voor zowel category 
management als beleidsmonitoring, met visualisaties van schapruimteverdeling, facing counts 
en labelzichtbaarheid.
\item Validatie van multimodale aanpak (beeld + OCR) zoals voorgesteld door Pettersson et 
al. \autocite{Pettersson2024} en Patel \autocite{Patel2025} in Belgische retailcontext.
\end{itemize}

\subsection{Impact voor doelgroep}
\label{subsec:impact}

De resultaten leveren tastbare meerwaarde op voor de volgende doelgroepen:

\begin{itemize}
\item \textbf{Category managers:} Objectieve data voor schapoptimalisatie, inzicht in 
positioning van gezonde producten, ondersteuning bij implementatie van blended 
meat-strategieën zoals toegepast door Colruyt, Lidl en Albert Heijn 
\autocite{GreenQueenBlendedMeat2024}.
\item \textbf{Beleidsmakers:} Empirisch bewijs voor regulering/ethische marketing, evaluatie 
Vlaamse eiwitstrategie (doelstelling $60\%$ plantaardig tegen 2030) 
\autocite{VlaanderenEiwitstrategie2023}, en monitoring van de $113$ lopende eiwitacties 
\autocite{VlaanderenEiwitVoortgang2025}.
\item \textbf{Retail consultants:} Schaalbare monitoringtool voor advies en A/B testen van 
schaplayouts, toepasbaar op de groeiende vraag naar plant-based producten in Europa ($6\%$ 
jaarlijkse groei tot 2027) \autocite{InnovaMarketInsights2025}.
\item \textbf{Consumentenbescherming:} Data ter ondersteuning van transparantie en ethische 
marketing, aansluitend bij consumentenvoorkeuren voor duurzaamheid ($67\%$ Belgen bereid meer 
te betalen) \autocite{MVOVlaanderenSustainapoll} en openheid voor alternatieve eiwitten 
($44\%$ geïnteresseerd in gekweekt vlees) \autocite{EuroconsumersCultivated2025}.
\end{itemize}

\section{Conclusie}
\label{sec:conclusie}

Dit onderzoek ontwikkelt een technisch robuuste, multimodale AI-pipeline voor 
\textbf{detectie, labelherkenning en spatial analysis} van plant-based producten in Belgische 
supermarkten. Het combineert state-of-the-art object detection (YOLOv8), OCR/template-based 
labelherkenning en nauwkeurige spatial metrics (facing counts, shelf share) op een gelabelde, 
regionale dataset. 

Verwachte resultaten omvatten reproduceerbare KPI's en visualisaties, statistische evaluaties 
van schappositie vs Nutri-Score, en praktische tools voor retailers en beleidsmakers. Hiermee 
levert het onderzoek zowel wetenschappelijke als maatschappelijke meerwaarde.

\printbibliography[heading=bibintoc]

\end{document}