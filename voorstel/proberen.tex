% !TeX program = xelatex
% !BIB program = biber

\PassOptionsToPackage{backend=biber,style=apa,sortcites=true,uniquename=false,maxbibnames=99}{biblatex}

\documentclass{hogent-article}

% --- STRUCTUUR/OPMAAK ---
\usepackage[protrusion=true,expansion=false]{microtype}
\emergencystretch=2em
\sloppy
\raggedbottom

% Laden van xcolor voor het markeren van de gevraagde secties
\usepackage{xcolor}
% Definieer de kleur donkerrood
\definecolor{DonkerRood}{rgb}{0.7, 0, 0}

\usepackage[dutch]{babel}
\usepackage{csquotes}
\DeclareLanguageMapping{dutch}{dutch-apa}

\addbibresource{voorstel.bib}

\usepackage{graphicx}
\graphicspath{{.}{../graphics/}} % Pad aangepast naar het hogere niveau (../graphics/)

\usepackage{caption}
\captionsetup{font=small,labelfont=bf}

\usepackage[section]{placeins}
\usepackage{float}

\usepackage{xurl}

\AtBeginDocument{\hypersetup{hidelinks}}

\usepackage{enumitem}
\setlist{topsep=3pt,itemsep=2pt,parsep=0pt,partopsep=0pt}

\clubpenalty=10000
\widowpenalty=10000
\displaywidowpenalty=10000

\setcounter{tocdepth}{2}

\hyphenation{
 klant-vragen
 gebruikers-account-analyse
 klanten-ser-vice
 door-ver-wij-zingen
 ver-wer-king
 ge-au-to-ma-ti-seer-de
}

% --- INFO OVER OPDRACHT ------------------------------------------------------
% Informatie over de opleiding, het vak en soort opdracht
\studyprogramme{Professionele bachelor toegepaste informatica}
\course{Research Methods}
\assignmenttype{Paper: Onderzoeksvoorstel}
\academicyear{2024--2025}

% TODO (fase 1): Werktitel
\title{Hoe kan een AI-gestuurd systeem de relatie analyseren tussen schappositie
en voedingswaarde (Nutri-Score A-E) van plant-based vleesvervangers en ontbijtgranen in Belgische supermarkten?}

% TODO (fase 1): Studentnaam en emailadres invullen
\author{Warre Vanpaemel}
\email{warre.vanpaemel@student.hogent.be}

% TODO (fase 1): Medestudent    
% Schrijf je het voorstel in samenwerking met een medestudent? Geef dan de naam
% en emailadres hier. Als je het voorstel alleen schrijft, verwijder dan deze
% regels of zet ze in commentaar.

% TODO (fase 1): Geef hier de link naar jullie Github-repository
\projectrepo{https://github.com/WarreVanp/latex-hogent-bachproef}

\specialisation{AI \& Data Engineering}
\keywords{AI, klantenservice, gebruikersanalyse, automatisering, NLP}

\begin{document}

\begin{abstract}
Dit onderzoek pakt het gebrek aan geautomatiseerde en objectieve methoden aan om de relatie 
tussen \textbf{schappositie} en \textbf{voedingswaarde} (Nutri-Score $A-E$) van plant-based producten
in Belgische supermarkten te analyseren. Voortbouwend op de Vlaamse eiwitstrategie 
(die streeft naar $60\%$ plantaardige eiwitconsumptie) en het feit dat ooghoogte het aankoopgedrag beïnvloedt, 
is de centrale vraag hoe een AI-gestuurd vision systeem deze relatie kan vaststellen. 
De doelstelling is de ontwikkeling van een end-to-end AI-pipeline, waarbij \textbf{YOLOv8} object detection en 
OCR (Optical Character Recognition) worden ingezet om producten en Nutri-Scores te herkennen op $100-150$ geannoteerde schapfoto's. 
Na classificatie in drie verticale zones (kinderooghoogte, volwassenenooghoogte, hoog), zal het systeem statistische correlaties tonen. 
De verwachte uitkomst is een systeem met minimaal $85\%$ detectienauwkeurigheid dat bewijs levert over 
de positionering van ongezonde producten (Nutri-Score $D/E$) op kinderooghoogte. Dit biedt supermarktketens een schaalbare monitoring-tool 
en geeft beleidsmakers objectieve data om ethische marketingrichtlijnen te evalueren.
\end{abstract}

\tableofcontents

\bigskip

% TODO: EP3

\section{Introductie}

Dit bacheloronderzoek richt zich op het ontwikkelen van een \textbf{AI-gestuurd computer systeem} 
om de schappositie van \textbf{plant-based vleesvervangers en ontbijtgranen} in Belgische supermarkten objectief
te analyseren en de relatie te onderzoeken met hun \textbf{voedingswaarde} (Nutri-Score $A-E$). De Vlaamse eiwitstrategie
2021--2030 streeft naar een verschuiving waarbij minstens $60\%$ van de eiwitinname plantaardig is. Tegelijkertijd 
breiden Belgische supermarktketens hun plant-based aanbod uit. Voor beleidsmakers en category managers is het cruciaal 
om inzicht te hebben in hoe producten gepositioneerd worden en of consumenten hierbij een voorkeur voor gezondere keuzes 
krijgen, bijvoorbeeld door gezonde producten op ooghoogte te plaatsen.

\subsection{Probleemstelling}

Handmatige analyses van schappen zijn tijdrovend, duur en subjectief. Er bestaat geen schaalbare methode om 
systematisch te meten welke producten op welke hoogte en met welke zichtbaarheid aanwezig zijn, noch hoe dit
samenhangt met de voedingswaarde. Dit vormt een belemmering voor:

\begin{itemize}
\item Category managers die schapindeling willen optimaliseren
\item Beleidsmakers die de effectiviteit van de Vlaamse eiwitstrategie willen monitoren
\item Retail consultants die evidence-based advies willen geven over schapindeling
\end{itemize}

subsection{Centrale onderzoeksvraag}
\label{subsec:onderzoeksvraag}

\begin{quote}
Hoe kan een AI-gestuurd systeem automatisch de relatie analyseren tussen schappositie 
(kinderooghoogte, volwassenenooghoogte, hoog) en voedingswaarde (Nutri-Score $A-E$) van 
plant-based vleesvervangers en ontbijtgranen in Belgische supermarkten?
\end{quote}

% (i) CONCRETE PROBLEEMSTELLING - Markering van de titel van de subsectie
\subsection*{\color{DonkerRood} Probleemstelling}

Handmatige analyses van schappen zijn tijdrovend, duur en subjectief. Er bestaat geen schaalbare methode om 
systematisch te meten welke producten op welke hoogte en met welke zichtbaarheid aanwezig zijn, noch hoe dit
samenhangt met de voedingswaarde. Dit vormt een belemmering voor:

\begin{itemize}
\item Category managers die schapindeling willen optimaliseren
\item Beleidsmakers die de effectiviteit van de Vlaamse eiwitstrategie willen monitoren
\item Retail consultants die evidence-based advies willen geven over schapindeling
\end{itemize}

% (ii) HOOFDONDERZOEKSVRAAG - Markering van de titel en de quote
\subsection*{\color{DonkerRood} Centrale onderzoeksvraag}
\label{subsec:onderzoeksvraag}

\begin{quote}
\color{DonkerRood}
Hoe kan een AI-gestuurd vision-systeem automatisch de relatie analyseren tussen schappositie 
(kinderooghoogte, volwassenenooghoogte, hoog) en voedingswaarde (Nutri-Score $A-E$) van 
plant-based vleesvervangers en ontbijtgranen in Belgische supermarkten?
\end{quote}

% (iii) DEELVRAGEN PROBLEEMDOMEIN - Markering van de titel en de tekst
\subsection*{\color{DonkerRood} Deelvragen probleemdomein}
\label{subsec:deelvragen-probleemdomein}

\begin{enumerate}
\item \color{DonkerRood} Wat zijn de huidige best practices in retail voor schapindeling en productpositionering 
van duurzame voedingsproducten, en welke metrics zijn relevant voor het beoordelen van 
schapindeling?

\item \color{DonkerRood} Welke uitdagingen bestaan er bij automatische productherkenning in 
supermarktomgevingen (wisselende lichtcondities, scheve hoeken, gedeeltelijk verborgen 
producten, dense packing)?

\item \color{DonkerRood} Wat zegt bestaand wetenschappelijk onderzoek over de relatie tussen schappositie 
(met name ooghoogte en kinderooghoogte) en consumentengedrag of verkoopvolume?
\end{enumerate}

% (iv) DEELVRAGEN OPLOSSINGSDOMEIN - Markering van de titel en de tekst
\subsection*{\color{DonkerRood} Deelvragen oplossingsdomein}
\label{subsec:deelvragen-oplossingsdomein}

\begin{enumerate}
\item \color{DonkerRood} Welke object detection algoritmes (YOLO, Faster R-CNN, EfficientDet) bereiken de 
hoogste nauwkeurigheid voor productdetectie op schapfoto's met variërende kwaliteit, en wat 
is de trade-off tussen nauwkeurigheid en inferentietijd?

\item \color{DonkerRood} Hoe kan een multimodaal AI-systeem (beeldherkenning + OCR) labels en productkenmerken 
automatisch classificeren en categoriseren, en hoe nauwkeurig kunnen duurzaamheidslabels 
specifiek worden herkend?

\item \color{DonkerRood} Welke beeldverwerkingstechnieken (augmentatie, normalisatie, perspective correction) 
verbeteren de detectienauwkeurigheid in realistische supermarktcondities met diverse 
belichtings- en hoeksituaties?

\item \color{DonkerRood} Hoe kunnen de gedetecteerde producten automatisch geanalyseerd worden op 
schapruimteverdeling, verticale positionering (kinderooghoogte vs volwassenenooghoogte vs 
hoog), en labelzichtbaarheid?

\item \color{DonkerRood} Welke statistische methoden (correlatie-analyse, chi-kwadraat test, regressie) kunnen 
de relatie tussen hoogtezone en Nutri-Score het best kwantificeren, en welke 
visualisatietechnieken maken de AI-output het meest interpreteerbaar voor retailers en 
beleidsmakers?
\end{enumerate}

\subsection{Doelstellingen}

De doelstellingen van dit onderzoek zijn:
\begin{enumerate}
\item Ontwikkelen van een end-to-end pipeline die producten detecteert, Nutri-Score herkent, 
verticale zones classificeert en correlaties berekent.
\item Opbouwen van een gelabelde dataset van Belgische schapfoto's met annotaties voor product, label en schapzone.
\item Het \textbf{benchmarking} en \textbf{fine-tunen} van de geselecteerde deep learning modellen 
(waaronder YOLOv8 en kleine CNN-classifiers) om robuuste product- en Nutri-Score-herkenning te garanderen 
onder diverse supermarktcondities.
\item Het \textbf{valideren} van de nauwkeurigheid en robuustheid van de gehele AI-pipeline aan de hand 
van een onafhankelijke set handmatige referentiedata en het vaststellen van de prestatie-indicatoren (zoals mAP en Nutri-Score F1-score).
\item Levering van praktische visualisaties en KPI's die bruikbaar zijn voor retailers en beleidsmakers.
\end{enumerate}

\section{State-of-the-Art}

\subsection{Context en technologische basis}

Recente literatuur toont een duidelijke toename van plant-based producten in Europese schappen, 
wat de behoefte aan objectieve monitoringtools vergroot. Productherkenning in de retailomgeving 
vereist een \textbf{multimodale benadering} waarbij zowel visuele kenmerken als tekstuele informatie worden 
gecombineerd om nauwkeurige onderscheidingen tussen \textbf{fine-grained} productcategorieën te maken \autocite{Pettersson2024, Patel2025}. 
Dit is essentieel omdat \textbf{dense packing}, kleine objecten en onregelmatige belichting in schappen grote 
uitdagingen vormen voor pure vision-systemen \autocite{Goldman2019}.

Voor de primaire detectietaak is de \textbf{YOLO-familie} (You Only Look Once), met name de nieuwere versie YOLOv8, 
de industriestandaard geworden. Deze biedt een optimale trade-off tussen snelheid en nauwkeurigheid, 
wat cruciaal is voor het real-time of batchgewijs verwerken van een groot volume aan schapfoto's \autocite{Redmon2024}. 
De robuuste herkenning van labels zoals Nutri-Score vereist verder een gespecialiseerde aanpak. Dit gebeurt door het 
combineren van generieke OCR-oplossingen met template matching of kleine CNN-classifiers die specifiek getraind zijn op de 
badges \autocite{Patel2025}. Terwijl \textbf{spatial analysis} (zoals facing counts en \textbf{eye-level premium}) 
momenteel meestal handmatig wordt uitgevoerd, zal dit onderzoek de schaarse methodes voor 
geautomatiseerde verticale positionering verder ontwikkelen \autocite{Goldman2019}.

\subsection{Onderzoeksproblemen en toegevoegde waarde}

Hoewel er diverse studies bestaan over algemene \textbf{grocery product recognition} datasets, 
is er een duidelijk gebrek aan onderzoek dat zich specifiek richt op \textbf{plant-based producten} in 
combinatie met de analyse van Nutri-Score en verticale schapposities binnen de Belgische supermarktcontext. 
Bestaande datasets missen vaak de lokale merkvariatie en het specifieke labelgebruik dat in deze regio 
dominant is. Dit onderzoek vult deze gebrekken door het combineren van een regionale dataset met een 
specifieke \textbf{focus op de correlatie} tussen verticale schappositie en voedingswaarde. 
De toegevoegde waarde ligt in de levering van een reproduceerbare, multimodale AI-pipeline en een 
statistische evaluatie van de associatie, resulterend in praktische KPI's die bruikbaar zijn voor zowel 
category management als beleidsanalyse.

\section{Methodologie}
\label{sec:methodologie}

De methodologie omvat zeven stappen waarin de deelvragen systematisch beantwoord worden via 
specifieke onderzoeksmethoden.

\subsection{Koppeling deelvragen aan onderzoeksmethoden}
\label{subsec:koppeling-methoden}

\textbf{Probleemdomein:}

\begin{itemize}
\item \textbf{Deelvraag 1} (Best practices schapindeling?) wordt beantwoord via 
\textit{literatuurstudie} van retail marketing papers en category management handboeken, met 
focus op eye-level premium en shelf positioning strategieën.

\item \textbf{Deelvraag 2} (Uitdagingen productherkenning?) wordt beantwoord via 
\textit{literatuurstudie} van state-of-the-art computer vision papers 
\autocite{Pettersson2024, Patel2025, Goldman2019}.

\item \textbf{Deelvraag 3} (Relatie schappositie en gedrag?) wordt beantwoord via 
\textit{literatuurstudie} van consumer behavior onderzoek en retail analytics studies.
\end{itemize}

\textbf{Oplossingsdomein:}

\begin{itemize}
\item \textbf{Deelvraag 1} (Welke object detection algoritmes?) wordt beantwoord via 
\textit{experimenteel onderzoek}: training en benchmarking van YOLOv8 en Faster R-CNN op de 
eigen dataset, met evaluatie op mAP@0.5, Precision, Recall en inferentietijd.

\item \textbf{Deelvraag 2} (Multimodaal systeem voor labels?) wordt beantwoord via 
\textit{experimenteel onderzoek}: implementatie van OCR (EasyOCR/Tesseract) + template 
matching voor Nutri-Score, validatie op 100 handmatig gelabelde producten, berekening van 
Accuracy en F1-score per labeltype.

\item \textbf{Deelvraag 3} (Beeldverwerkingstechnieken?) wordt beantwoord via 
\textit{experimenteel onderzoek}: systematisch testen van augmentaties en 
preprocessing technieken, vergelijking van detectienauwkeurigheid met/zonder deze technieken.

\item \textbf{Deelvraag 4} (Automatische spatial analysis?) wordt beantwoord via 
\textit{ontwikkeling en implementatie} van algoritmes voor verticale zoneclassificatie, 
facing counting en shelf share berekening, met \textit{validatie} tegen handmatige metingen 
(afwijking in \%).

\item \textbf{Deelvraag 5} (Statistische methoden en visualisatie?) wordt beantwoord via 
\textit{statistische analyse}: chi-kwadraat test, Kruskal-Wallis ANOVA, Pearson/Spearman 
correlatie op verzamelde data, en \textit{ontwikkeling} van dashboard met heatmaps, KPI's en 
geannoteerde foto's.
\end{itemize}


De eerste fase richt zich op data. Dit begint met de \textbf{Data-acquisitie}, waarbij $100-150$ hoge-resolutie foto's 
worden verzameld van relevante schappen in $3-5$ verschillende supermarkten om voldoende variatie in belichting 
en lay-out te garanderen. Essentiële metadata, zoals filiaal, schaplocatie en tijdstip, wordt vastgelegd. 
Dit wordt onmiddellijk gevolgd door een intensieve \textbf{Annotatie}-fase. Bounding boxes worden handmatig 
getrokken voor elk product, Nutri-Score label en andere relevante labels (Bio, Vegan) met behulp van tools zoals CVAT. 
Cruciaal hierbij is het labelen van de productcategorie en de schapzone.

Zodra de dataset is gelabeld, verschuift de focus naar modelontwikkeling. De \textbf{Object Detection}-stap 
gebruikt \textbf{YOLOv8} als primaire detector, aangevuld met een benchmark tegen een traditionele detector 
zoals Faster R-CNN. De modellen worden geoptimaliseerd door \textit{fine-tuning} op de eigen dataset, 
het gebruik van \textit{augmentaties}, en de implementatie van een \textit{tiling-strategie} om 
kleine objecten beter te kunnen detecteren. Parallel daaraan vindt de \textbf{Label Recognition} plaats. Dit 
vereist een multi-staps benadering: eerst tekstextractie via generieke OCR (EasyOCR/Tesseract), gevolgd door 
gespecialiseerde template matching of een kleine CNN-classifier om de Nutri-Score badges te lokaliseren en 
classificeren.

De output van de detectie- en herkenningsmodellen geven vervolgens de \textbf{Spatial Analysis}. Dit is de 
meest kritieke fase van de analyse: op basis van de gedetecteerde coördinaten wordt de verticale zone 
geclassificeerd (kinderooghoogte, volwassene, hoog). Hieruit worden nauwkeurige \textbf{facing counts}, het \textbf{shelf share} (percentage schapruimte),
en een label zichtbaarheidsscore berekend.

De laatste fasen omvatten de \textbf{Analyse} en de presentatie. Voor de analyse worden de verkregen metingen 
statistisch getoetst (chi-kwadraat, ANOVA, regressie) om de correlatie tussen schaphoogte en voedingswaarde te bepalen. 
De resultaten van de AI-pipeline worden gevalideerd tegen onafhankelijke handmatige \textbf{ground-truth} metingen om de 
nauwkeurigheid vast te stellen. Tot slot volgt de \textbf{Visualisatie}: de resultaten worden gepresenteerd in statische grafieken
en tabellen waarin de vastgenomen bevindingen weergegeven worden.

\subsection{Tijdsschatting (modulair)}

\begin{enumerate}
\item Data-acquisitie en annotatie: $4$ weken.
\item Modelontwikkeling (YOLOv8 detectie, label recognition, augmentatie): $4$ weken.
\item Spatial analysis en validatie: $2$ weken.
\item Visualisatie en dashboard: $1$ weken.
\item Rapportage en conclusie: $2$ weken.
\end{enumerate}

\section{Verwachte resultaten}

De belangrijkste verwachte resultaten zijn de oplevering van:
\begin{itemize}
\item Een volledig \textbf{reproduceerbare AI-pipeline} voor productdetectie, Nutri-Score herkenning en schapanalyse.
\item Gedetailleerde \textbf{benchmarkresultaten} die de nauwkeurigheid versus de inferentietijd van YOLOv8 en eventuele andere geteste modellen (zoals Faster R-CNN) documenteren.
\item Een \textbf{statistische correlatieanalyse} die het verband tussen verticale schappositie en Nutri-Score empirisch onderbouwt, inclusief odds ratios en betrouwbaarheidsintervallen.
\item Een praktisch en bruikbaar \textbf{KPI-dashboard} dat dient voor zowel category management als beleidsmonitoring.
\end{itemize}

\subsection{Impact voor doelgroep}

De resultaten leveren tastbare meerwaarde op voor de volgende doelgroepen:
\begin{itemize}
\item \textbf{Category managers:} Zij ontvangen objectieve, schaalbare data voor het optimaliseren 
van de schapindeling en krijgen inzicht in de daadwerkelijke positionering van gezonde producten.
\item \textbf{Beleidsmakers:} Zij verkrijgen empirisch bewijs ter onderbouwing van regulering/ethische
 marketing en voor de evaluatie van de effectiviteit van de Vlaamse eiwitstrategie.
\item \textbf{Retail consultants:} Zij kunnen de AI-pipeline gebruiken als een schaalbare monitoring-tool 
voor advies en A/B testen van schaplayouts.
\end{itemize}

\section{Conclusie}

Dit onderzoek ontwikkelt een technisch robuuste, multimodale AI-pipeline 
voor \textbf{detectie, labelherkenning en spatial analysis} van plant-based producten in Belgische supermarkten. 
Het combineert state-of-the-art object detection (YOLOv8), OCR/template-based labelherkenning en 
nauwkeurige spatial metrics (facing counts, shelf share) op een gelabelde, regionale dataset. 
Verwachte resultaten omvatten reproduceerbare KPI's en visualisaties, statistische evaluaties 
van schappositie vs Nutri-Score, en praktische tools voor retailers en beleidsmakers. Hiermee 
levert het onderzoek zowel wetenschappelijke als maatschappelijke meerwaarde.

\begin{thebibliography}{9}

% Hier gebruiken we \bibitem met de sleutels uit de tekst:

\bibitem{Pettersson2024}
Pettersson, T., Riveiro, M., \& L{\"o}fstr{\"o}m, T. (2024). \textit{Multimodal fine-grained grocery product recognition using image and {OCR} text}. Machine Vision and Applications, \textit{35}(79).

\bibitem{Patel2025}
Patel, S. R. (2025). \textit{Multi-modal product recognition in retail environments: Enhancing accuracy through integrated vision and {OCR} approaches}. World Journal of Advanced Research and Reviews, \textit{25}(1), 1837--1844.

\bibitem{Goldman2019}
Goldman, E., Herzig, R., Eisenschtat, A., Goldberger, J., \& Hassner, T. (2019). \textit{Precise Detection in Densely Packed Scenes}. In \textit{Proceedings of the IEEE/CVF Conference on Computer Vision and Pattern Recognition (CVPR)}, 5227--5236.

\bibitem{Redmon2024}
Ultralytics. (2024). \textit{YOLOv8 Documentation}. Beschikbaar via \url{https://docs.ultralytics.com/}.

\bibitem{BryantSanctorumc2021}
Bryant, C., \& Sanctorum, H. (2021). \textit{Alternative proteins, evolving attitudes: Comparing consumer attitudes to plant-based and cultured meat in Belgium in two consecutive years}. Appetite, \textit{161}(105161).

\end{thebibliography}

\end{document}