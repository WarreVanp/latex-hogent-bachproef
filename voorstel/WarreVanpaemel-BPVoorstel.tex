%==============================================================================
% Sjabloon onderzoeksvoorstel bachproef
%==============================================================================
% Gebaseerd op document class `hogent-article'
% zie <https://github.com/HoGentTIN/latex-hogent-article>

% Voor een voorstel in het Engels: voeg de documentclass-optie [english] toe.
% Let op: kan enkel na toestemming van de bachelorproefcoördinator!
\documentclass{hogent-article}

% Invoegen bibliografiebestand
\addbibresource{voorstel.bib}

% Informatie over de opleiding, het vak en soort opdracht
\studyprogramme{Professionele bachelor toegepaste informatica}
\course{Bachelorproef}
\assignmenttype{Onderzoeksvoorstel}
% Voor een voorstel in het Engels, haal de volgende 3 regels uit commentaar
% \studyprogramme{Bachelor of applied information technology}
% \course{Bachelor thesis}
% \assignmenttype{Research proposal}

\academicyear{2025--2026} % TODO: pas het academiejaar aan

% TODO: Werktitel
\title{Hoe kan een AI-gestuurd vision systeem de relatie analyseren tussen schappositie 
en voedingswaarde (Nutri-Score) van plant-based vleesvervangers en ontbijtgranen in Belgische supermarkten?}

% TODO: Studentnaam en emailadres invullen
\author{Warre Vanpaemel}
\email{warre.vanpaemel@student.hogent.be}

% TODO: Geef de co-promotor op
\supervisor[Co-promotor]{/}
% S. Beekman (Synalco, \href{mailto:sigrid.beekman@synalco.be}{sigrid.beekman@synalco.be})
% Binnen welke specialisatierichting uit 3TI situeert dit onderzoek zich?
% Kies uit deze lijst:
%
% - AI \& Data Engineering
%
\specialisation{AI \& Data Engineering}
\keywords{Computer Vision, Object Detection, Nutri-Score, YOLOv8, Shelf Positioning, Product Recognition, Spatial Analysis}

\begin{document}

\begin{abstract}
In het kader van de Vlaamse eiwitstrategie 2021--2030, die streeft naar 60\% plantaardige eiwitconsumptie, 
en de toenemende beschikbaarheid van plant-based vleesvervangers en gezondere ontbijtgranen in Belgische supermarkten,
komt de vraag of de schappositie van deze producten correleert met hun voedingswaarde. 
Uit retailonderzoek blijkt dat producten op ooghoogte, en specifiek op kinderooghoogte, een grote invloed hebben op aankoopgedrag, 
maar er ontbreekt objectieve, geautomatiseerde methoden om de relatie tussen schappositie en 
gezondheidswaarde (Nutri-Score) systematisch te analyseren. Dit onderzoek beantwoordt de vraag: 
hoe kan een AI-gestuurd vision systeem de relatie analyseren tussen schappositie (kinderooghoogte, volwassenenooghoogte, hoog) 
en voedingswaarde (Nutri-Score A-E) van plant-based vleesvervangers en ontbijtgranen in Belgische supermarkten? 
De doelstelling is het ontwikkelen van een end-to-end AI-pipeline die automatisch producten detecteert via YOLOv8 
object detection, Nutri-Score labels herkent door middel van template matching en OCR, 
de verticale schappositie classificeert in drie hoogtzones, en een statische weergave van de gevonden correlaties weergeeft. 
De methodologie omvat het fotograferen en annoteren van 100--150 schapfoto's in 3--5 supermarktfilialen, 
het trainen van deep learning modellen op deze dataset, het implementeren van spatial analysis algoritmes voor 
hoogteclassificatie, en het valideren van de AI-nauwkeurigheid tegen handmatige metingen.
Verwacht wordt dat het systeem een detectienauwkeurigheid van minimaal 85\% bereikt en aantoont of er een 
significante correlatie bestaat tussen schappositie en Nutri-Score, specifiek of ongezonde producten (Nutri-Score D/E) 
disproportioneel op kinderooghoogte gepositioneerd worden. De meerwaarde voor supermarktketens ligt in een schaalbare 
monitoring-tool voor category management en naleving van ethische marketing-richtlijnen, terwijl beleidsmakers 
objectieve data krijgen om de effectiviteit van de Vlaamse eiwitstrategie te evalueren en eventuele regulering 
van schappositie voor ongezonde producten te onderbouwen. 
\end{abstract}

\tableofcontents

% De hoofdtekst van het voorstel zit in een apart bestand, zodat het makkelijk
% kan opgenomen worden in de bijlagen van de bachelorproef zelf.
\input{voorstel-inhoud}

\printbibliography[heading=bibintoc]

\end{document}